% ==============================================================================
\subsection{Small Perturbation Model}
We get the linearised model using small perturbtation:
\begin{align*}
    J \delta \dot \omega + b_m \delta \omega + 2 C_D \omega_0 \delta \omega  &= \delta u_\omega \lr{V_{in} b_m + 2V_{in}^2 C_D u_{\omega_0}}\\
    J \delta \dot \omega + (b_m + 2 C_D \omega_0) \delta \omega  &= \delta u_\omega \lr{V_{in} b_m + 2V_{in}^2 C_D u_{\omega_0}}\\
    \text{Laplace Transform:} \qquad \qquad &\\
     \lr{Js + (b_m + 2 C_D \omega_0)} \delta \omega  &= \delta u_\omega \lr{V_{in} b_m + 2V_{in}^2 C_D u_{\omega_0}}
\end{align*}
Thus we have the trannsfer function model:
\begin{align*}
    \frac{\delta \omega(s)}{\delta u_\omega (s)} &= \frac{V_{in} b_m + 2V_{in}^2 C_D u_{\omega_0}}{Js + (b_m + 2 C_D \omega_0)}
\end{align*}

Thus both gain and time-constant increase with the nominal rpm. Getting the gain and cut-off frequency using the standard first-order model:
\begin{align*}
    \frac{\delta \omega(s)}{\delta u_w(s)} &= \frac{k_m}{s + \omega_m} = \frac{V_{in} b_m + 2V_{in}^2C_D u_{\omega_0}}{Js + (b_m + 2 C_D \omega_0)}\\
    \text{Where, } \qquad &\\
    k_p &= \frac{V_{in}}{J} \lr{ b_m + 2V_{in}C_D u_{\omega_0}}\\
    \omega_p &= \frac{1}{J}\lr{b_m + 2 C_D \omega_0}
\end{align*}

This information will be used for establishing the validity of identified model.

In case of using $u_p$ as input:
\begin{align*}
    u_\omega &= a u_p + b\\
    \implies \delta u_\omega &= a \delta u_p\\
    \implies \frac{\delta \omega(s)}{\delta u_p(s)} &= \frac{1}{a}\frac{k_p}{s + \omega_p}
\end{align*}

Thus this results in variation of the static gain alone.
