Intoducing the input definition into the BLDC-motor model with propeller:
\begin{align*}
    J\dot \omega + b_m \omega + C_D \omega^2 + M_f &= u K_r V_{in} = K_r V_{in} g_\omega(u_\omega, \hat V_{in})\\
    \implies  J \dot \omega + b_m \omega + C_D \omega^2 + M_f &= K_r V_{in} \lr{\frac{b_m}{K_r} u_\omega + \frac{\hat V_{in}}{K_r} C_D u_\omega^2 + \frac{M_f}{K_r  \hat V_{in}}}\\
    J \dot \omega + b_m \omega + C_D \omega^2 + M_f \lr{1 - \frac{V_{in}}{\hat V_{in}}} &= V_{in} b_m u_\omega + V_{in} \hat V_{in} C_D u_\omega^2
\end{align*}

\itbf{Note on Voltage:} The battery voltage is assumed to be constant with small variations that can be introduced as uncertainities.
\begin{align*}
    \hat V_{in} &= V_{in} ( 1 + \delta v)
    \implies \frac{V_{in}}{\hat V_{in}} = 1 - \delta v
    \implies \lr{1 - \frac{V_{in}}{\hat V_{in}}} = \delta v
\end{align*}
\begin{equation}
    J \dot \omega + b_m \omega + C_D \omega^2 + M_f \delta v = V_{in} b_m u_\omega + V_{in}^2 (1 + \delta v) C_D u_\omega^2
\end{equation}

%===============================================================================
\subsection{Parametric Identification model for BLDC motor - propeller dyanmcis}
We have:
\begin{equation*}
    J \dot \omega + b_m \omega + C_D \omega^2 + M_f \delta v = V_{in} b_m u_\omega + V_{in}^2 (1 + \delta v) C_D u_\omega^2
\end{equation*}
