\subsection{ESC and non-linear input compensation}

The castle creations ESC has a micro-controller that non-lineraly scales the input PWM signal's duty-cycle to the duty-cycle of the $24\,kHz -$PWM signals to the inverter effectively scaling the source voltage to the motor by the duty-cycle. This is done to have a linear input to thrust curve instead of a quadratic one.

The input transmission can be described as follows: The $400\,Hz-$PWM duty cycle ($u_p$) islineraly scaled to a throttle ratio $(p)$ ($\%$ power-out) between 0 and 1. Which is then filters using a non-linear function $(g_u)$ to get the PWM duty-cycle input to the inverter ($u$) \cite{kim2017electric}.

\begin{align*}
    u_p \rightarrow \boxed{g_u(.)} \rightarrow u
\end{align*}
and finally,
$$V_s = u V_{in} \qquad u \in [0, 1]$$

$u$ is considered as the input to the motor-propeller system and the necessary invertion will be performed for transmitting the signals.

\itbf{Scaling PWM Singal Duty cycle based on switching frequency}
Pixhawk-4 uses a switching frequency of $400\,Hz$ for its PWM signals (can be swithed to $50 \, Hz$ which is not that usefull in case of BLDC motors but usefull for servos). The controller thus scales the PWM duty cycle to the duration of on-time of the signal in its period in 'microseconds'. These inputs are handelled as integer types within the range [800, 2200]\cite{px4_pwm}.

The current ESC that has the rpm-feedback capabilites has an operating range between $1110 \, \mu s$ and $1890 \, \mu s$. After that, the ESC switchs to a constant power mode which sets the rpm to a constant.
\begin{align*}
    \text{Period of the PWM wave } &= \frac{1}{400} \times 10^6 \, \mu s = 2500 \, \mu s\\
    \text{Minimum Operating Duty Cycle } &= \frac{1110}{2500} = 0.444\\
    \text{Maximum Operating Duty Cycle } &= \frac{1890}{2500} = 0.756\\
\end{align*}

$u$ can be considered to be the actual input to the system and system identification with the propeller. It turns out that the parameters of the above non-linear filter are not estimatable with the give information. To solve this problem, we chose normalized no-load angular velocity of the motor as the input instead.
