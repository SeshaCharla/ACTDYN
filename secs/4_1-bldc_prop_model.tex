Intoducing the input definition into the BLDC-motor model with propeller:
\begin{align*}
    & J\dot \omega + b_m \omega + C_D \omega^2 + M_f = u K_r V_{in} = K_r V_{in} g_\omega(u_\omega, \hat V_{in})\\
    \implies & J \dot \omega + b_m \omega + C_D \omega^2 + M_f =K_r V_{in} \lr{\frac{b_m}{K_r} u_\omega + \frac{\hat V_{in}}{K_r} C_D u_\omega^2 + \frac{M_f}{K_r  \hat V_{in}}}\\
\end{align*}
\begin{equation}
    J \dot \omega + b_m \omega + C_D \omega^2 + M_f \lr{1 - \frac{V_{in}}{\hat V_{in}}} = V_{in} b_m u_\omega + V_{in} \hat V_{in} C_D u_\omega^2\\
\end{equation}


\subsection{Small Perturbation Model}
We get the linearised model using small perturbtation:
\begin{align*}
    J \delta \dot \omega + b_m \delta \omega + 2 C_D \omega_0 \delta \omega  &= \delta u_\omega \lr{V_{in} b_m + V_{in} \hat V_{in} C_D u_{\omega_0}}\\
    J \delta \dot \omega + (b_m + 2 C_D \omega_0) \delta \omega  &= \delta u_\omega \lr{V_{in} b_m + V_{in} \hat V_{in} C_D u_{\omega_0}}\\
    \text{Laplace Transform:} \qquad &\\
     \lr{Js + (b_m + 2 C_D \omega_0)} \delta \omega  &= \delta u_\omega \lr{V_{in} b_m + V_{in} \hat V_{in} C_D u_{\omega_0}}
\end{align*}
Thus we have the trannsfer function model:
\begin{align*}
    \frac{\delta \omega(s)}{\delta u_\omega (s)} &= \frac{V_{in} b_m + V_{in} \hat V_{in} C_D u_{\omega_0}}{Js + (b_m + 2 C_D \omega_0)}
\end{align*}

Thus both gain and time-constant increase with the nominal rpm.
\begin{align*}
    \frac{\delta \omega(s)}{\delta u_w(s)} &= \frac{k_p}{s + \omega_p} = \frac{\frac{(K_rK_v + b_f) V_{in}}{J}}{s + \frac{(K_rK_v + b_f + 2 C_D \omega_0)}{J}}\\
    \text{Where, } \qquad &\\
    k_p &= \frac{(K_rK_v + b_f) V_{in}}{J} & (\text{Independent of } \omega_0)\\
    \omega_p &= \frac{(K_rK_v + b_f + 2 C_D \omega_0)}{J}
\end{align*}

This information will be used for establishing the validity of identified model.

In case of using $u_p$ as input:
\begin{align*}
    \frac{\delta \omega(s)}{\delta u_p(s)} &= \frac{g'_w (u_p) k_p}{s + \omega_p} = \frac{\left(\frac{22 k_p}{u_p - 1110}\right)}{s + \omega_p}
\end{align*}

Thus in this case the numerator gain varies inversely with $u_{p_{nom}}$ and $\omega_p$ varies with $\omega_0$.

%===============================================================================
\subsection{Parametric Identification model for BLDC motor - propeller dyanmcis}
Let, $J$ be the total moment on inertia along the normal axis. Then,
\begin{align*}
    J \dot \omega &= -(K_rK_v + b_f) \omega_m - M_f - C_D \omega_m^2 + u K_r V_{in}\\
   %===
    \implies J\dot \omega &= -(K_rK_v + b_f) \omega - C_D \omega^2 - M_f + V_{in} \left((K_r K_v  + b_f) u_\omega + \frac{M_f}{\hat V_{in}} \right)\\
   %===
        &= -(K_rK_v + b_f) \omega - C_D \omega^2 - M_f \left( 1 - \frac{V_{in}}{\hat V_{in}}\right) + V_{in} \left((K_r K_v  + b_f) u_\omega  \right)\\
    %==
    \implies J \dot \omega &= -(K_rK_v + b_f)\omega_m - C_D \omega_m^2 - M_f \left( 1 - \frac{V_{in}}{\hat V_{in}}\right) + u_\omega (K_rK_v + b_f) V_{in}
    %===
\end{align*}
$$\boxed{J \dot \omega + (K_rK_v + b_f)\omega + C_D \omega_m^2 + M_f \left( 1 - \frac{V_{in}}{\hat V_{in}}\right) =  u_\omega (K_rK_v + b_f) V_{in}}$$

As $M_m = (K_rK_v + b_f) \omega + C_D \omega^2 + M_f \left( 1 - \frac{V_{in}}{\hat V_{in}}\right) $ is measured, we use it in the right-hand side of parameter esitmation to reduce the number of parameters.

Descritizing the above equation:
\begin{align*}
    \frac{J}{h} (\omega[k+1] - \omega[k]) - u_\omega[k] (K_rK_v + b_f) V_{in} &= - M_m \\
    %===
    J(\omega[k+1] - \omega[k]) - u_\omega[k] h(K_rK_v + b_f) V_{in} &= - M_m h \\
    %===
    \underbrace{\bm{(\omega[k+1] - \omega[k]) & -h u_\omega[k]} }_{\phi(\omega[k+1], \omega[k], u[k])} \underbrace{\bm{J \\ (K_rK_v + b_f) V_{in} }}_{\pmb \theta} &= - M_m h
\end{align*}



% Hence, we have the continuous parametric model:
% $$ \dot \omega =
%     \underbrace{\begin{bmatrix}
%     - \omega^2 & - \omega  & -1
% \end{bmatrix}}_{\Phi^T(\omega)}
% \underbrace{\begin{bmatrix}
%     \frac{C_{D}}{J} \\
%     \frac{(K_rK_v + b_f)}{J}  \\
%     M_f \left( 1 - \frac{V_{in}}{\hat V_{in}}\right)
% \end{bmatrix} }_{\pmb \theta}
%     +\left(\frac{(K_rK_v + b_f)}{J} V_{in}\right) u_{\omega}
%  $$

% Let,
% \begin{align*}
%     a_1 = \frac{C_{D}}{J}  \qquad
%     a_2 &= \frac{(K_rK_v + b_f)}{J} \qquad
%     a_3 =   M_f \left( 1 - \frac{V_{in}}{\hat V_{in}}\right)\\
%     \\
%     \pmb \theta = \begin{bmatrix}a_1 & a_2 & a_3 \end{bmatrix}^T &\qquad
%     b = \frac{(K_rK_v + b_f)}{J} V_{in} = a_2 V_{in}
% \end{align*}

% $$\therefore \dot \omega = \Phi^T(\omega) \pmb \theta + b u$$

% \subsubsection{Descritezed Parametric Model}
% Descritizing the above model using euler-method $\left(\dot \omega = \frac{\omega[k] - \omega[k-1]}{h}\right)$, with $h$ as the sampling interval:

% \begin{align*}
%     \frac{\omega[k] - \omega[k-1]}{h} &=
%     \begin{bmatrix} - \omega^2[k-1] & - \omega[k-1]  &  -1 \end{bmatrix}
%     \begin{bmatrix}
%         a_1 \\ a_2 \\ a_3
%     \end{bmatrix}
%     + bu[k-1]\\
%     %===
%     \omega[k] &= h \begin{bmatrix} - \omega^2[k-1] & - \omega[k-1]  & -1 \end{bmatrix}
%     \begin{bmatrix}
%         a_1 \\
%         a_2 - \frac{1}{h} \\
%         a_3 \\
%     \end{bmatrix}
%     + b  h u[k-1]\\
%     %===
%     &= h \begin{bmatrix} - \omega^2[k-1] & -\omega[k-1] & -1 & u[k] \end{bmatrix}
%     \begin{bmatrix}
%         a_1 \\
%         a_2 - \frac{1}{h} \\
%         a_3 \\
%         b
%     \end{bmatrix}
% \end{align*}

% Let,
% \begin{align*}
%     \pmb \theta_h = h
%     \begin{bmatrix}
%         a_1 \\
%         a_2 - \frac{1}{h} \\
%         a_3 \\
%         b
%     \end{bmatrix} =
%     h
%     \begin{bmatrix}
%         \frac{C_{D}}{J} \\
%         \frac{(K_rK_v + b_f)}{J} - \frac{1}{h}  \\
%         M_f \left( 1 - \frac{V_{in}}{\hat V_{in}}\right)\\
%         \frac{(K_rK_v + b_f)}{J} V_{in}
%     \end{bmatrix}
%     \qquad \text{and} \qquad
%     \Phi(\omega[k-1], u[k])^T &=  \begin{bmatrix} - \omega^2[k-1] & -\omega[k-1] & -1 & u[k-1] \end{bmatrix}
% \end{align*}

% hence, we have the parametric model in least-squares form:
% \begin{align*}
%     \omega[k] &= \Phi(\omega[k-1], u[k-1])^T \pmb \theta_h
% \end{align*}

%\subsubsection{Input singal (persistance of exitation and frequency limitation)}
%\textbf{Note:}
%\begin{enumerate}
%    \item PE order of a square wave of half-period m is m+1.
%    \item PE order of a single sine wave is 2.
%\end{enumerate}
%Hence, a sum of sinusoids wave with atleast 3 waves within the frequency of 45 Hz (the limitation is due to the structure) can be used to estimate the parameters and the coefficients of force and torque generated.
%
