%===============================================================================
\subsection{Non-linear Model Parameter Identification}
We have:
\begin{equation*}
    J \dot \omega + b_m \omega + C_D \omega^2 + M_f \delta v = V_{in} b_m u_\omega + V_{in}^2 (1 + \delta v) C_D u_\omega^2
\end{equation*}



% Hence, we have the continuous parametric model:
% $$ \dot \omega =
%     \underbrace{\begin{bmatrix}
%     - \omega^2 & - \omega  & -1
% \end{bmatrix}}_{\Phi^T(\omega)}
% \underbrace{\begin{bmatrix}
%     \frac{C_{D}}{J} \\
%     \frac{(K_rK_v + b_f)}{J}  \\
%     M_f \left( 1 - \frac{V_{in}}{\hat V_{in}}\right)
% \end{bmatrix} }_{\pmb \theta}
%     +\left(\frac{(K_rK_v + b_f)}{J} V_{in}\right) u_{\omega}
%  $$

% Let,
% \begin{align*}
%     a_1 = \frac{C_{D}}{J}  \qquad
%     a_2 &= \frac{(K_rK_v + b_f)}{J} \qquad
%     a_3 =   M_f \left( 1 - \frac{V_{in}}{\hat V_{in}}\right)\\
%     \\
%     \pmb \theta = \begin{bmatrix}a_1 & a_2 & a_3 \end{bmatrix}^T &\qquad
%     b = \frac{(K_rK_v + b_f)}{J} V_{in} = a_2 V_{in}
% \end{align*}

% $$\therefore \dot \omega = \Phi^T(\omega) \pmb \theta + b u$$

% \subsubsection{Descritezed Parametric Model}
% Descritizing the above model using euler-method $\left(\dot \omega = \frac{\omega[k] - \omega[k-1]}{h}\right)$, with $h$ as the sampling interval:

% \begin{align*}
%     \frac{\omega[k] - \omega[k-1]}{h} &=
%     \begin{bmatrix} - \omega^2[k-1] & - \omega[k-1]  &  -1 \end{bmatrix}
%     \begin{bmatrix}
%         a_1 \\ a_2 \\ a_3
%     \end{bmatrix}
%     + bu[k-1]\\
%     %===
%     \omega[k] &= h \begin{bmatrix} - \omega^2[k-1] & - \omega[k-1]  & -1 \end{bmatrix}
%     \begin{bmatrix}
%         a_1 \\
%         a_2 - \frac{1}{h} \\
%         a_3 \\
%     \end{bmatrix}
%     + b  h u[k-1]\\
%     %===
%     &= h \begin{bmatrix} - \omega^2[k-1] & -\omega[k-1] & -1 & u[k] \end{bmatrix}
%     \begin{bmatrix}
%         a_1 \\
%         a_2 - \frac{1}{h} \\
%         a_3 \\
%         b
%     \end{bmatrix}
% \end{align*}

% Let,
% \begin{align*}
%     \pmb \theta_h = h
%     \begin{bmatrix}
%         a_1 \\
%         a_2 - \frac{1}{h} \\
%         a_3 \\
%         b
%     \end{bmatrix} =
%     h
%     \begin{bmatrix}
%         \frac{C_{D}}{J} \\
%         \frac{(K_rK_v + b_f)}{J} - \frac{1}{h}  \\
%         M_f \left( 1 - \frac{V_{in}}{\hat V_{in}}\right)\\
%         \frac{(K_rK_v + b_f)}{J} V_{in}
%     \end{bmatrix}
%     \qquad \text{and} \qquad
%     \Phi(\omega[k-1], u[k])^T &=  \begin{bmatrix} - \omega^2[k-1] & -\omega[k-1] & -1 & u[k-1] \end{bmatrix}
% \end{align*}

% hence, we have the parametric model in least-squares form:
% \begin{align*}
%     \omega[k] &= \Phi(\omega[k-1], u[k-1])^T \pmb \theta_h
% \end{align*}

\subsubsection{Input singal (persistance of exitation and frequency limitation)}
\textbf{Note:}
\begin{enumerate}
   \item PE order of a square wave of half-period m is m+1.
   \item PE order of a single sine wave is 2.
\end{enumerate}
%Hence, a sum of sinusoids wave with atleast 3 waves within the frequency of 45 Hz (the limitation is due to the structure) can be used to estimate the parameters and the coefficients of force and torque generated.
%
